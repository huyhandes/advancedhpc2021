\documentclass{article}
\usepackage[utf8]{inputenc}
\usepackage{listings}

\title{Labwork 3}
\author{BUI Quang Huy}
\date{October 2021}

\begin{document}

\maketitle

Running labwork 3:
\begin{itemize}
    \item Copy the memory from inputImage to devInput(device memory)
    \item Run the rgb2grayCUDA kernel
    \item Copy the memory from devGray(device memory) to outputImage
    \begin{lstlisting}
        __global__ void rgb2grayCUDA(uchar3 *input, uchar3 *output) {
        int tid = threadIdx.x + blockIdx.x * blockDim.x;
        output[tid].x = (input[tid].x + input[tid].y + input[tid].z) / 3;
        output[tid].z = output[tid].y = output[tid].x;
    }

void Labwork::labwork3_GPU() {
    // Calculate number of pixels
    int pixelCount = inputImage->width * inputImage->height * 3;  // let's do it 100 times, otherwise it's too fast!
    uchar3 *devInput;
    uchar3 *devGray;
    outputImage = static_cast<char *>(malloc(pixelCount));
    // Allocate CUDA memory
    // Copy CUDA Memory from CPU to GPU
    cudaMalloc(&devInput, pixelCount * sizeof(uchar3));
    cudaMalloc(&devGray, pixelCount * sizeof(uchar3));
    cudaMemcpy(devInput, inputImage->buffer, pixelCount, cudaMemcpyHostToDevice);

    // Processing
    int blockSize = 512;
    int numBlock = pixelCount / (blockSize*3);
    rgb2grayCUDA<<<numBlock, blockSize>>>(devInput, devGray);
    
    // Copy CUDA Memory from GPU to CPU
    cudaMemcpy(outputImage, devGray, pixelCount, cudaMemcpyDeviceToHost);
    // Cleaning
    cudaFree(devInput);
    cudaFree(devGray);
}
    \end{lstlisting}
    \item Compile make -j
    \item Run command ./labwork 3 ../data/cloud.jpg
    \item Result:\begin{itemize}
        \item labwork 3 ellapsed 2.9ms
    \end{itemize}
\end{itemize}
\end{document}
